\chapter*{Resumen}

Este trabajo constituye la Tesis de Grado necesaria para obtener el titulo de
Ingeniero Electrónico de la Facultad de Ingeniería de la Universidad de Buenos
Aires.

El trabajo se realizó en el marco de las investigaciones llevadas a cabo en el
Centro de Simulación Computacional (CSC) del CONICET. En particular, este
trabajo se encuentra enmarcado en las investigaciones referentes a aplicaciones
de tecnología UWB en la caracterización y detección de objetos y materiales.

El objetivo de la tesis es el diseño, implementación y caracterización de un
generador de pulsos ultracortos. Este tiene aplicaciones en los caminos de
recepción y transmisión de sistemas UWB, habilitando la simplificación y mejora
de desempeño de los mismos. Para esto se desarrollará un circuito generador de
pulsos y una etapa driver que permite su fácil integración en sistemas UWB
usuales.

Se abordarán los fundamentos teóricos que sustentan el funcionamiento del
circuito diseñado, seguido por un análisis exhaustivo mediante simulaciones que
serán validadas con mediciones experimentales. Para concluir, se ofrecerán
perspectivas para futuras investigaciones que puedan aprovechar los cimientos
establecidos en esta tesis.
