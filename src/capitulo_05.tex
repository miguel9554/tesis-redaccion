\chapter{Conclusiones}

\section{Conclusiones}

En el presente trabajo se desarrolló exitosamente un generador de pulsos
ultracortos. El mismo logró amplitudes de pulso de entre \qty{400}{\milli\volt}
y \qty{1.1}{\volt} y FHWM de entre \qty{159}{\pico\second} y
\qty{165}{\pico\second}, que es equivalente a un ancho de banda a \qty{3}{\dB}
de \qty{2.8}{\giga\hertz} y a \qty{10}{\dB} de \qty{4.5}{\giga\hertz}.  Estos
pulsos fueron generados usando una alimentación $V_{cc}$ de entre \qty{5}{\volt}
y \qty{7}{\volt}, comandando la generación de pulsos con la salida digital de
una FPGA, de \qty{3.3}{\volt} y baja capacidad de carga. Se estudió la
aplicabilidad de los pulsos generados en una cadena de transmisión simplificada
para un sistema UWB obteniendo resultados satisfactorios.

En cuanto al generador diseñado, el mismo resultó extremadamente simple,
compuesto únicamente por un diodo SRD, un diodo Schottky con la función de
rectificación y una línea de transmisión cortocircuitada a tierra proveyendo el
control del ancho del pulso. Para lograr el comando del generador mediante una
interfaz digital de baja capacidad de carga y la generación de pulsos en base a
una única fuente unipolar, se incorporó una etapa \textit{driver}, también de
extremado bajo costo. Esta etapa se construyó en base a un integrado
\textit{gate driver} seguido de un capacitor serie de filtrado pasalto para
generar tensión negativa, siendo el \textit{gate driver} el único dispositivo
activo utilizado.

Para el circuito diseñado se presentaron múltiples expresiones relacionando los
valores de los componentes con los parámetros del circuito, expresiones que
fueron validadas mediante simulaciones y mediciones. Estas expresiones incluyen
la amplitud de pulso, la duración temporal y el consumo de corriente del
circuito. Todo el circuito fue validado previo a la fabricación y medición
mediante simulaciones, utilizando uno de los modelos reportados en la literatura
sobre diodos SRD.

Respecto al diseño físico, se fabricaron dos PCB, uno para la etapa driver y
otro para la etapa pulser. Esto le proporciona versatilidad al trabajo
desarrollado, permitiendo intercambiar estas placas con otros posibles
desarrollos futuros. El diseño de las mismas estuvo acompañado de simulaciones
electromagnéticas previas que permitieron optimizar la impedancia de las líneas
de transmisión en toda la banda de frecuencias de interés. Esto resultó en una
mínima degradación en las mediciones obtenidas con respecto a las predicciones
realizadas por los modelos de simulación circuital.

Comparando los resultados obtenidos con los reportados en la literatura, se
obtuvieron parámetros de pulso comparables con el estado del arte en generadores
UWB discretos. En particular, el presente trabajo resalta por el funcionamiento
en base a una única fuente de alimentación unipolar, por el extremado bajo costo
evidenciado por la baja cantidad de dispositivos utilizados y por la posibilidad
de ser controlado directamente por la salida digital de una FPGA o
microcontrolador.

\section{Trabajos a futuro}

Como trabajo a futuro, es posible desarrollar las aplicaciones en caminos de
recepción y transmisión de sistemas UWB para el generador desarrollado. En el
caso de la transmisión, en este trabajo se presentó una simulación de nivel
esquemático demostrando la aplicabilidad del trabajo desarrollado en la tarea.
En el caso del receptor, sería necesario primero llegar a una simulación
esquemático del circuito cómo se hizo en este trabajo con el camino de
transmisión. En ambos casos el diseño físico es desafiante, en especial para el
receptor, dado que necesita un generador de pulsos con salida diferencial, por
lo que sería necesario desarrollar un balun UWB.

En el caso del desarrollo de las aplicaciones del generador de pulsos, se
refinarían los requerimientos sobre los pulsos. La plataforma de transmisión
tendría requisitos más exigentes en cuanto a \textit{ringing} para mejorar la
calidad de la señal transmitida, mientras que la de recepción
mayores requisitos de amplitud de pulso para mejorar el rango dinámico de
conversión. Dada la modularidad y simplicidad del generador desarrollado, es
posible iterar sobre su diseño para mejorar las prestaciones que eventualmente
requieran estas aplicaciones.

En cuanto al modelado, en las mediciones se observaron efectos que no fueron
predichos por los modelos de simulación disponibles. En particular, se
observaron oscilaciones y un segundo pulso de menor magnitud. Es de interés
explorar el origen de estas discrepancias, que puede ser tanto por un modelado
incorrecto de los componentes parásitos del circuito cómo por limitaciones del
modelo utilizado para el SRD. Una vez identificada la causa de estas
observaciones, es posible incorporarlas a los modelos de simulación utilizados,
y una vez reproducidas estas no-idealidades en una simulación, es posible la
exploración de modificaciones al generador con el objetivo de mitigar estos
efectos.
