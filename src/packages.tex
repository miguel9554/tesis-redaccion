\usepackage[spanish, es-tabla]{babel}
\usepackage[utf8]{inputenc}
\usepackage{pdfpages}
\usepackage{amsmath}
\usepackage{xcolor}
\usepackage{graphicx}
\usepackage{a4wide}
\usepackage{float}
\usepackage{hyperref,xcolor}
\usepackage{siunitx}
\usepackage{caption}
\usepackage{listings}
\usepackage{threeparttable}
\usepackage{pgfplots}
\usepackage{titlesec}
\usepackage[nottoc,numbib]{tocbibind}
\usepackage[american]{circuitikz}
\usepackage{subcaption}
\usepackage{multirow}
\usepackage{fancyhdr}
\usepackage{adjustbox}


% Para mostrar "Código" en lugar de "Listing"
\renewcommand{\lstlistingname}{Código}

% From https://www.overleaf.com/learn/latex/Code_listing#Code_styles_and_colours
\definecolor{codegreen}{rgb}{0,0.6,0}
\definecolor{codegray}{rgb}{0.5,0.5,0.5}
\definecolor{codepurple}{rgb}{0.58,0,0.82}
\definecolor{backcolour}{rgb}{0.95,0.95,0.92}

\lstdefinestyle{PythonStyle}{
    backgroundcolor=\color{backcolour},
    commentstyle=\color{codegreen},
    keywordstyle=\color{magenta},
    numberstyle=\tiny\color{codegray},
    stringstyle=\color{codepurple},
    basicstyle=\ttfamily\footnotesize,
    breakatwhitespace=false,
    breaklines=true,
    captionpos=b,
    keepspaces=true,
    numbers=left,
    numbersep=5pt,
    showspaces=false,
    showstringspaces=false,
    showtabs=false,
    tabsize=2
}

% Customizaciones para listing de verilog
\lstdefinestyle{VerilogStyle}{
  language=Verilog,
  basicstyle=\ttfamily\small,
  keywordstyle=\color{blue}\bfseries,
  commentstyle=\color{green!60}\itshape,
  stringstyle=\color{orange},
  identifierstyle=\color{black},
  showstringspaces=false,
  breaklines=true
}

% Unidades custom
\DeclareSIUnit{\mil}{mil}
\DeclareSIUnit{\voltRMS}{V_{RMS}}
\DeclareSIUnit{\dBm}{dBm}

% Declaramos la función matemática "erf" para graficarla más adelante
\makeatletter
\pgfmathdeclarefunction{erf}{1}{%
  \begingroup
    \pgfmathparse{#1 > 0 ? 1 : -1}%
    \edef\sign{\pgfmathresult}%
    \pgfmathparse{abs(#1)}%
    \edef\x{\pgfmathresult}%
    \pgfmathparse{1/(1+0.3275911*\x)}%
    \edef\t{\pgfmathresult}%
    \pgfmathparse{%
      1 - (((((1.061405429*\t -1.453152027)*\t) + 1.421413741)*\t
      -0.284496736)*\t + 0.254829592)*\t*exp(-(\x*\x))}%
    \edef\y{\pgfmathresult}%
    \pgfmathparse{(\sign)*\y}%
    \pgfmath@smuggleone\pgfmathresult%
  \endgroup
}
\makeatother
