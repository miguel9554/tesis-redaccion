\chapter{Introducción}

Para los requerimientos, hablar en algún lado de la plataforma base? Por ahora
citamos \cite{altieri2017}

Poner en algún lado una tabla con todos los requisitos: ancho de banda, amplitud, PRF, entrada de comando, fuente de alimentación, etc

\section{Introducción a UWB}

La tecnología UWB (del inglés \textit{Ultra Wide Banda}, ultra ancho de banda) es una
tecnología de radio caracterizada por señales y sistemas cuyos anchos de banda son muy
grandes. En el dominio del tiempo, estos grandes anchos de banda se traducen en
señales ultra cortas. Esto es contraposición a los sistemas de RF usuales, que
suelen ser de banda angosta o sintonizados. Las señales temporales ultracortas o
el ancho de banda grande supone un desafío para el diseño de los sistemas.

La Comisión Federal de Comunicaciones de los Estados Unidos, conocida como la
FCC, es la agencia gubernamental encargada de regular las comunicaciones en el
país. Su misión principal es supervisar y establecer normativas para una amplia
gama de servicios de comunicación, que incluyen radio, televisión, telefonía,
Internet y otros medios de comunicación. La FCC juega un papel crucial en la
asignación de espectro de radiofrecuencia, la promoción de la competencia en el
mercado de las comunicaciones y la protección de los consumidores. La influencia
global de la FCC se debe al gran mercado que representan los Estados Unidos en
el ámbito de las comunicaciones y tecnologías de la información. Dado que muchas
empresas y productos de tecnología de la comunicación se desarrollan en los
Estados Unidos y tienen un alcance global, los estándares y regulaciones
establecidos por la FCC a menudo se convierten en referencia para otros países y
regiones, lo que asegura una cierta uniformidad y compatibilidad en el mercado
global, facilitando la interconexión de dispositivos y servicios en todo el
mundo.

Según la FCC (\textit{Federal Communications Commission}), un sistema UWB se
define como uno en el que el ancho de banda utilizado es mayor a
\qty{500}{\mega\hertz} o \qty{20}{\percent} de la frecuencia portadora
\cite{FCC_UWB}. Para la definición de ancho de banda, la comisión utiliza la
definición de ancho de banda a \qty{10}{\dB}, en lugar de los \qty{3}{\dB}
usuales en otros campos. De esta manera el ancho de banda queda definido como la
franja de frecuencia en la que la potencia cae \qty{10}{\dB} con respecto al
punto de máxima potencia.

En las tecnologías de radio de banda angosta, que componen la mayoría de las
asignadas por la FCC, se asigna un canal angosto a cada aplicación. Debido al
gran ancho de banda de los sistemas UWB, es inevitable que el rango de
frecuencias se superponga con alguno asignado a otra tecnología. Para evitar
interferencia excesiva, sobre todo con sistemas críticos como el GPS, la FCC
asigna una máscara espectral sobre la potencia máxima permitida para los
sistemas UWB. En la figura \ref{fig:fcc_uwb_psd_mask} puede observarse la misma.
La máscara se establece sobre la PIRE, Potencia Isotrópica Radiada Equivalente
(o EIRP del inglés \textit{Equivalent Isotropic Radiated Power}). Esta cantidad
es la potencia que debería emitir un radiador isotrópico para realizar la misma
intensidad de radicación en la dirección de máxima potencia de la antena. Es una
medida de direccionalidad de la antena que permite comparar entre radiadores con
distintos patrones de radiación.

\begin{figure}[t]
    \centering
    \includegraphics[width=0.8\textwidth]{images/fcc_uwb_psd_mask.png}
    \caption{Máscara de PSD para señales UWB establecida por la FCC. Tomada de \cite{Heydari2005}.}
    \label{fig:fcc_uwb_psd_mask}
\end{figure}

La tecnología UWB encuentra aplicaciones en diversas areas

\begin{itemize}
  \item \textbf{Radar:} El gran ancho de banda de las señales UWB se presta
        para aplicaciones de radar que requieran de gran resolución espacial. Su
        capacidad de penetración profunda permite aplicaciones de radar en las
        que se pueden resolver objetos bloqueados por paredes. Su gran
        penetración también permite aplicaciones en GPR (del inglés
        \textit{Ground Penetrating Radar}, radar de penetración de suelo), en
        las que la señal de radar se utiliza para identificar materiales
        enterrados en el suelo o dentro de muros. Algunos ejemplos
        pueden consultarse en \cite{morales2018}, \cite{savelyev2010},
        \cite{senapati2021}.
  \item \textbf{Comunicaciones:} Otra ventaja del gran ancho de banda de estos
        sistemas es la capacidad de transmisión de datos que puede obtenerse.  En
        sistemas de transmisión de datos de corto alcance, pueden explotarse las
        propiedades de las señales UWB para obtener altas tasas de transmisión.
        Algunos ejemplos son \cite{jaesang2004}, \cite{zhiquan2005},
        \cite{Heydari2005}
  \item \textbf{Imágenes médicas:} Existen aplicaciones en las que radares de
        impulsos Doppler UWB son utilizados para medición de signos vitales
        críticos como pulso cardíaco y respiración. Este tipo de radares frente
        a otros de onda continua presentan ventajas como menor consumo y mayor
        resolución. Algunos ejemplos son \cite{jalivand2011}, \cite{oloumi2020},
        \cite{jalivand2011_2}.
  \item \textbf{Caracterización de materiales:} ciertos materiales pueden ser
      caracterizados mediante su iluminación con una señal UWB y un análisis
        sobre las reflexiones de la señal. El gran ancho de banda de esta
        respuesta permite realizar caracterizaciones del material que con
        señales de banda angosta no serían posibles. Algunos ejemplos de estas
        aplicaciones pueden consultarse en \cite{altieri2017},
        \cite{Salman2008}, \cite{Bouza2023}, \cite{Muqaibel2003},
        \cite{salman2010performance}, \cite{Altieri2021}
\end{itemize}

En cuanto a su adopción, en 2019 varios telefonos inteligentes empezaron a
ofrecer tecnologías UWB. En campos de radar fue expicada su aplicación.

Habiendo sido presentadas las ventajas de los sistemas UWB y como estas permiten
su aplicación en distintos campos, estos tienen desventajas en cuanto a
complejidad de implementación. El gran ancho de banda de las señales permite
explotar diversas aplicaciones pero al mismo tiempo presenta complicaciones en
el diseño, ya que los caminos de señal deben tener un gran ancho de banda. Esto
vuelve desafiante el diseño en varios aspectos.

El objetivo de este trabajo es desarrollar un generador de pulsos UWB de bajo
costo, que evite estas problemáticas.

\section{Generadores de pulsos ultracortos}

\textcolor{red}{Agregar referencias acá}

Un componente fundamental de los sistemas UWB son los generadores de pulsos
ultracortos. Estos tienen múltiples aplicaciones, tanto en la generación de las
señales a ser transmitidas, como en muestro y conversión a banda
base/pasante. Es fundamental en este tipo de generadores lograr anchos de pulso
menores a \qty{1}{\nano\second} y tener amplitudes variables.

Existen diversas topologías para la implementación de estos circuitos. En la
literatura se encuentran resultados reportados con implementaciones tanto
integradas como discretas. En los generadores implementados en circuitos
integrados, son usuales las arquitecturas basadas en la suma de una señal de
entrada con distintos niveles de retardo, generado pulsos con forma de gaussiana
derivada \cite{Nguyen2012} \cite{Salehi2010} \cite{An2018}.Como en este trabajo
el objetivo es una plataforma de bajo costo y reconfigurable, se optará por una
con una implementación discreta.

En la familia de generadores implementados con componentes discretos, destacan
los implementados con diodos SRD. A continuación se resaltarán las principales
características de los mismos, las distintas topologías existentes y sus
ventajas y desventajas.

\subsection{Generadores basados en diodo SRD}

Un tipo de generadores de pulsos ultracortos son los basados en diodo SRD. Este
tipo de dispositivos pertenecen a la familia de diodos de almacenamiento de
carga. Se diferencian por su característica de recuperación reversa. Esta
presenta un largo tiempo de almacenamiento seguido de un tiempo de transición
muy corto, lo que permite su aplicación en generación de pulsos ultracortos.

Este tipo de generadores de pulsos se caracterizan por su bajo costo,
versatilidad en cuanto a amplitud y ancho de pulso, y baja complejidad de
implementación. Es por estos motivos que son ampliamente utilizados en la
generación de pulsos para sistemas UWB.

Existen distintas arquitecturas de generación de pulsos basados en este diodo.
En todas se explota su característica de recuperación reversa, las variaciones
se presentan en el tipo de componentes utilizados, el tipo de señal de entrada y
su acople y la forma de utilización del diodo. En la mayor parte de los
generadores reportados en la literatura, se puede diferenciar entre SRD serie o
paralelo con la señal, y la utilización de inductores o líneas de transmisión en
paralelo, o \textit{stub}, para la generación del pulso.

Separarémos a los generadores en dos grupos: serie y paralelo, explicando las
ventajas y desventajas de cada uno.

\begin{figure}
    \centering
    \begin{subfigure}[b]{0.45\textwidth}
        \centering
        \includegraphics[width=\textwidth]{images/srd_series_generator.png}
        \caption{Generador SRD serie. Tomado de \cite{han2005}.}
        \label{fig:srd_series_generator}
    \end{subfigure}
    \hfill
    \begin{subfigure}[b]{0.45\textwidth}
        \centering
        \includegraphics[width=\textwidth]{images/srd_shunt_generator.png}
        \caption{Generador SRD paralelo. Tomado de \cite{han2005}.}
        \label{fig:srd_shunt_generator}
    \end{subfigure}
    \caption{Generadores de pulsos basados en SRD con topología serie y
    paralelo.}
    \label{fig:srd_pulse_generator_topologies}
\end{figure}

\subsubsection{Serie}

En los generadores que denominaremos serie, el diodo SRD se encuentra en serie
con la señal. En la figura \ref{fig:srd_series_generator} se observa un ejemplo.
En este caso, el diodo SRD genera un flanco muy rápido en base a uno
(posiblemente) lento, Con este flanco rápido, se utiliza un stub cortocircuitado
a tierra para reflejarlo y así lograr un pulso ultra corto. La duración temporal
de este pulso estará dado por el retardo de propagación en la línea de
transmisión, mientras que la amplitud estará determinada por el retardo en la
línea, la amplitud de la señal de entrada y la velocidad de crecimiento del
flanco generado por el SRD. La señal de entrada puede estar acoplada de manera
directa al diodo, o acoplada en alterna. El primer tipo de acople tiene asociado
una mayor disipación de potencia, mientras que la segunda presenta menor
disipación pero más complejidad de implementación, ya que la correcta
polarización del diodo y su transición al estado de alta impedancia se vuelven
más desafiantes.

Estos generadores se caracterizan por su simplicidad de implementación. El hecho
de utilizar una línea de transmisión para la generación del pulso simplifica el
diseño, ya que no requiere la adquisición de componentes adicionales, siendo que
la selección de los mismos es desafiante en el contexto de UWB dónde se debe
trabajar en grandes anchos de banda. La utilización de la línea de transmisión
permite más control sobre el diseño.

\subsubsection{Paralelo}

En los generadores paralelo, el diodo SRD se encuentra en paralelo con la
salida. El SRD presenta una baja impedancia mientras se encuentra polarizado en
directa. Una vez aplicada una corriente inversa para conmutar al dispositivo a
su estado de alta impedancia, el mismo permanece en el estado de baja impedancia
durante un tiempo denominado tiempo de almacenamiento. Extinguido este período,
el diodo transiciona al estado de alta impedancia en un tiempo dado por su
tiempo de transición, que para diodos SRD es del orden de picosegundos. Teniendo
al diodo en paralelo con la salida, es posible generar un pulso ultra corto en
base a esta transición. Para lograr esto, se acopla capacitivamente el SRD a la
salida, con un valor de capacidad que filtre toda la forma de onda excepto la
transición rápida. De esta manera se conforma el generador de pulsos.

Es usual en estos generadores la presencia de un inductor serie con la señal de
entrada, como puede observarse en la figura \ref{fig:srd_shunt_generator}.

también durante una porción de tiempo denominada tiempo de almacenamiento
De esta manera, la señal de salida es la tensión sobre el diodo SRD.
Para formar el pulso, se acopla la salida capacitivamente, de manera que la
salida es la tensión sobre el SRD acoplada en alterna. 
Estos generadores suelen disponer de un inductor en serie con la señal
para lograr la generación del pulso. En la figura \ref{fig:srd_shunt_generator}
se observa el esquemático.

\section{Aplicación de generadores de pulsos ultracortos}

A continuación se detallará la aplicación de generadores de pulsos ultracortos
en el contexto de un sistema UWB. Se utilizará de referencia el trabjo reportado
en \cite{Altieri2021}, donde se desarrolló una plataforma para la estimación del
contenido de humedad en poliamidas en base a UWB, es decir, una aplicación de
caracterización de materiales. En esta plataforma se irradia sobre el objeto a
caracterizar una señal UWB, cuya reflexión es capturada por la misma plataforma
y analizada para estimar el contenido de humedad. La plataforma contiene
entonces dos caminos, el de transmisión y el de recepción, el diagrama en
bloques de ambos puede observarse en la figura
\ref{fig:uwb_system_block_diagram}. A continuación se detallará como un
generador de pulsos ultracortos permite simplificar la implementación de estos
bloques.

\begin{figure}[t]
    \centering
    \begin{subfigure}[b]{0.45\textwidth}
        \centering
        \includegraphics[width=\textwidth]{images/uwb_system_tx_path.png}
        \caption{Camino de transmisión en plataforma UWB. Tomado de
        \cite{Altieri2021}.}
        \label{fig:uwb_system_tx_path}
    \end{subfigure}
    \hfill
    \begin{subfigure}[b]{0.45\textwidth}
        \centering
        \includegraphics[width=\textwidth]{images/uwb_system_rx_path.png}
        \caption{Camino de recepción en plataforma UWB. Tomado de
        \cite{Altieri2021}.}
        \label{fig:uwb_system_rx_path}
    \end{subfigure}
        \caption{Cadenas de transmisión y recepción en sistema UWB reportado en
        \cite{Altieri2021}.}
    \label{fig:uwb_system_block_diagram}
\end{figure}

\subsection{Aplicaciones en transmisión}

El diagrama en bloques del camino de transmisión puede observarse en la figura
\ref{fig:uwb_system_tx_path}. La función de esta cadena es generar un pulso en
banda pasante a ser transmitido por la antena, que irradiará sobre el objeto a
caracterizar. En este caso, se toma una frecuencia central de transmisión de
\qty{1.9}{\giga\hertz} y un ancho de banda de \qty{1}{\giga\hertz}. Para la
transmisión de este pulso en banda pasante, se parte de un pulso generado por
una FPGA. Este pulso tiene un ancho mínimo dado por el reloj mínimo del sistema,
que en este caso se encontraba en \qty{1.5}{\nano\second}, con un ancho de banda
de \qty{10}{\dB} de \qty{1.28}{\giga\hertz}. Este pulso se encuentra en banda
base, para pasarlo a banda pasante se utiliza un multiplicador de frecuencia
pasivo. Para la multiplicación es necesaria una señal de referencia de
\qty{1.9}{\giga\hertz}, y dado que el multiplicador es pasivo, este requiere que
la potencia de la referencia sea de \qty{13}{\dBm}.  Para lograr esto se
requiere de un oscilador activo y un amplificador para lograr la potencia
necesaria. Dado que el multiplicador es pasivo y tiene una pérdida especificada
en \qty{10}{\dB}, es necesario un bloque de ganancia para compensar. En este
caso, se utilizaron dos bloques de ganancia intercalados por un atenuador
configurable, de manera de darle a todo el sistema una ganancia configurable de
entre \qty{-13.4}{\dB} y \qty{18.1}{\dB}.

A continuación se explicará cómo esta cadena puede ser fuertemente simplificada
con un generador de pulsos ultracortos. En la figura
\ref{fig:proposed_uwb_tx_path} se observa la cadena de señal propuesta. Para
generar el pulso en banda pasante, en lugar de utilizar una arquitectura basada
en una generación de pulso en banda base y luego una modulación a banda pasante,
se utilizará una basada en un generador de pulsos y un filtro pasabanda. El
pulso en banda pasante de la arquitectura original se encontraba en una
frecuencia central de \qty{1.9}{\giga\hertz} con un ancho de banda de
\qty{1}{\giga\hertz}. Para generar este pulso, se propone una arquitectura en la
que se dispone de un filtro pasabanda con las características del pulso deseado,
es decir, una frecuencia central de \qty{1,9}{\giga\hertz} y
\qty{1}{\giga\hertz} de ancho de banda, y se lo excita con un pulso ultra corto.
Este pulso aproxima a un impulso ideal, siendo que excitar un sistema LTI con un
impulso ideal resulta en una señal con un espectro igual a la función
transferencia del sistema, en este caso la salida tendrá el espectro del filtro
pasabanda. Por ser el ancho del pulso finito y distinto de 0, la salida no será
exactamente igual a la transferencia del filtro, pero con un pulso lo
suficientemente corto es suficiente. En terminos de frecuencia, es suficiente
con que el ancho de banda a \qty{10}{\dB} sea mayor a la máxima frecuencia útil,
en este caso $ \qty{1,9}{\giga\hertz}+ \qty{1}{\giga\hertz}/2 =
\qty{2,4}{\giga\hertz}$. Para mantener la funcionalidad de ganancia variable,
entre el pulser y el filtro pasabanda se incluyen un amplificador y un atenuador
variable.

De esta manera, se implementa una cadena que requería de 6 componentes por una
que solo requiere de 2. En el caso de la arquitectura propuesta, la potencia de
salida puede ser ajustada con la amplitud del generador de pulsos, ya que la
misma define la amplitud del pulso en banda pasante. Regulando la amplitud de
este generador se regula la potencia. En cuanto a la forma espectral del pulso
de salida, es totalmente controlable por el filtro pasabanda, por lo que la
cadena propuesta presenta una gran versatilidad, además de baja complejidad.
Cabe resaltar la importancia del generador en esta cadena, que debe generar
pulsos ultracortos (tales que su ancho de banda sea mayor al de los pulsos en
banda pasante) y amplitud regulable. En caso de ser necesario, pueden agregarse
un amplificador y atenuador variable como se indica en la figura
\ref{fig:proposed_uwb_tx_path} para darle más versatilidad al ajuste de
amplitud.

Además de la menor cantidad de componentes en la arquitectura propuesta, estos
son más simples. En la arquitectura original se necesita, además de los
amplificadores y el atenuador variable, un oscilador y un multiplicador, ambos
componentes con significativas complejidades de implementación y limitaciones
asociadas.

\begin{itemize}
    \item El oscilador requiere un diseño de una placa especial, y el diseño de
        la misma es de complejidad ya que el camino del  oscilador es crítico en
        el desempeño del sistema. También requiere de un control digital, ya sea
        en un microcontrolador, con el costo asociado al desarrollo del
        programa. Además, la salida es de baja potencia, lo que requiere un
        amplificador adicional.
    \item En cuanto al multiplicador, dado el gran ancho de banda de las señales
        de trabajo, es necesario utilizar uno pasivo. Estos son dispositivos no
        lineales, en general de menor desempeño que los activos adaptados a
        señales de banda angosta.  Estos multiplicadores requieren portadoras de
        potencia (\qty{10}{\dBm}-\qty{15}{\dBm}) considerable que el oscilador
        por defecto no alcanza, por lo que es necesario un amplificador. Ademas
        tienen restricciones del ancho de banda en banda base, y restricciones
        del ancho de banda en banda pasante; además no deben superponerse las
        bandas. Por ejemplo, en el HMC213 utilizado en la plataforma de
        referencia \cite{Altieri2021}, la banda de entrada es hasta
        \qty{1.5}{\giga\hertz}, y la banda pasante es de
        \qty{1.5}{\giga\hertz}-\qty{4.5}{\giga\hertz}. Entonces si se quiere
        pasar una señal de \qty{1}{\giga\hertz} en banda base, al modularla
        queda con \qty{2}{\giga\hertz} en banda pasante, la portadora mínima
        debe ser \qty{2.5}{\giga\hertz} para evitar que se superponga la banda
        de entrada con la banda pasante).
\end{itemize}

En la arquitectura propuesta, no es necesario el multiplicador y, por lo tanto,
tampoco el oscilador. Removiendo estos componentes se elimina gran parte del
costo monetario y de ingeniería de la plataforma, ya que como fuese descripto
anteriormente, estos dos componentes involucran múltiples restricciones y
complejidades en la implementación. En la cadena de transmisión propuesta, la
mayor complejidad está en la implementación del pulser. Una vez diseñado este
componente, la etapa de transmisión solamente necesita el diseño del filtro
pasabanda, que puede ser realizado directamente con una línea de transmisión en
la placa, y opcionalmente la inclusión de un amplificador comercial.

\begin{figure}[t]
    \centering
    \begin{subfigure}[b]{0.45\textwidth}
        \centering
        \includegraphics[width=\textwidth]{images/proposed_uwb_tx_path.drawio.png}
        \caption{Propuesta de camino de transmisión en plataforma UWB basado en
        generador de pulsos ultra cortos.}
        \label{fig:proposed_uwb_tx_path}
    \end{subfigure}
    \hfill
    \begin{subfigure}[b]{0.45\textwidth}
        \centering
        \includegraphics[width=\textwidth]{images/proposed_uwb_rx_path.drawio.png}
        \caption{Propuesta de camino de recepción en plataforma UWB basado en
        generador de pulsos ultra cortos.}
        \label{fig:proposed_uwb_rx_path}
    \end{subfigure}
    \caption{Propuestas de cadenas de transmisión y recepción en sistema UWB
    basado en generador de pulsos ultracortos.}
    \label{fig:uwb_system_block_diagram}
\end{figure}

\subsection{Aplicaciones en recepción}

En cuanto al receptor, en la figura \ref{fig:uwb_system_rx_path} se observa la
cadena de señal implementada en la plataforma de referencia. La misma está
compuesta por un amplificador de bajo ruido como primer etapa, seguido de un
filtro pasabanda antialiasing. Luego del filtrado se realiza una conversión a
banda base con un demodulador I/Q, seguido de una amplificación. Finalmente se
realiza una conversión A/D con un conversor de tiempo real con tasa de muestreo
de \qty[per-mode=symbol]{1.8}{\giga\siemens\per\second}. Este conversor de tiempo
real vuelve muy costoso al sistema, ya que un conversor de tan alta tasa de
muestreo tiene un ancho de banda analógico muy grande, lo que vuelve el costo
tanto monetario como de desarrollo de ingeniería muy alto.

Explotando la periodicidad de la señal recibida, es posible simplificar esta
cadena reemplazándola por una basada en un generador de pulsos ultracortos,
realizando en lugar de muestreo en tiempo real un muestreo en tiempo
equivalente, técnica posibilitada por la periodicidad de la señal de entrada. En
este esquema, se muestrea a una tasa mucho menor a la de Nyquist, muestreando la
señal en distintos puntos de su período. Luego, estos puntos pueden ser
alineados correctamente para reconstruir la señal original. En la figura
\ref{fig:Illustration_of_equivalent_time_sampling} puede observarse el esquema.
Se observa como se toman muestras a una tasa mucho menor a la de Nqyuist, con el
cuidado de en cada muestra tomar un punto distinto dentro de la forma de onda a
muestrear. De esta manera, el conversor en tiempo real de gran tasa de muestreo
puede ser reemplazado por un conversor de baja tasa y bajo costo, reduciendo de
manera importante el costo del sistema.

Con el muestreo en tiempo equivalente es posible muestrear una señal a una tasa
mucho menor a la de Nyquist. Sin embargo, un conversor A/D de la baja tasa de
muestreo deseada tiene un ancho de banda analógico bajo, por lo que no sería
posible muestrear directamente el pulso en banda pasante. Para hacer de interfaz
entre esta señal y el conversor de baja tasa, es posible implementar un circuito
de \textit{Sample \& Hold} que muestrea la señal, proveyendo al conversor una
señal discretidad de bajo ancho de banda. En la figura
\ref{fig:sampling_circuit} se observa un circuito de muestreo basado en pulsos
ultra cortos reportado en \cite{han2004}. El mismo se basa en diodos para
muestrear la señal, que son accionados por pulsos ultracortos. La duración
temporal de estos pulsos define el ancho de banda máximo de entrada. La amplitud
determina el rango dinámico.

Con este esquema basado en muestreo en tiempo equivalente e implementado con un
circuito de muestreo basado en diodos de muestreo accionados por pulsos ultra
cortos, es posible reemplazar al costoso conversos de tiempo real de la figura
\ref{fig:uwb_system_rx_path} por el circuito propuesto. De esta manera, se llega
a una cadena de recepción como la de la figura \ref{fig:proposed_uwb_rx_path}.
En este sistema, se mantienen como primeras etapas el amplificador de bajo ruido
y el filtro pasabanda. Estos componentes son de bajo costo y de baja complejidad
de implementación. Luego, en lugar del demodulador I/Q, los amplificadores
diferenciales y el conversor de Nyquist, se encuentra el circuito de muestreo
excitado por el pulser. Los pulsos de control del circuito de muestreo
trabajarían a una frecuencia de repetición PRF (del inglés \textit{Pulse
Repetition Frequency}) de \qty{10}{\mega\hertz}. A la salida del muestreador se
realizaría una forma de onda compuesta por las muestras del pulso en banda
pasante. Esta señal es amplificada por un amplificador operacional de bajo ancho
de banda y muestreada por un ADC de baja tasa, siendo la necesaria de
\qty[per-mode=symbol]{10}{\mega\siemens\per\second}, más de 2 ordenes de
magnitud por debajo de la tasa del conversor de la figura
\ref{fig:uwb_system_rx_path}.

\begin{figure}[t]
    \centering
    \begin{subfigure}[b]{0.45\textwidth}
        \centering
        \includegraphics[width=\textwidth]{images/Illustration-of-equivalent-time-sampling.png}
        \caption{Ilustración de la técnica de muestreo en tiempo equivalente.}
        \label{fig:Illustration_of_equivalent_time_sampling}
    \end{subfigure}
    \hfill
    \begin{subfigure}[b]{0.45\textwidth}
        \centering
        \includegraphics[width=\textwidth]{images/sampling_circuit.png}
        \caption{Circuito de muestreo basado en pulsos ultra cortos. Tomado de
        \cite{han2004}.}
        \label{fig:sampling_circuit}
    \end{subfigure}
    \caption{Esquema de muestreo en tiempo equivalente y circuito de muestreo
    reportado en \cite{han2004}.}
    \label{fig:equivalent_time_sampling_figures}
\end{figure}

\section{Requerimientos del generador de pulsos}

En base a las posibles aplicaciones en los caminos de transmisión y recepción de
pulsos de una plataforma UWB, se determina el siguiente conjunto de
especificaciones para el generador de pulsos.

\begin{table}
\centering
\begin{tabular}{c|c}
\hline
    Variable & Requerimiento \\
\hline
    $V_{in}$                &   CMOS @ $V_{DD}=\qty{3.3}{\volt}$ ($V_{OH}$
    \qty{2.4}{\volt})     \\
    PRF                &        \qty{10}{\mega\hertz} \\
    $V_{dd}$                &   \qty{5}{\volt} - \qty{8}{\volt} \\
    $A$                &        \qty{500}{\milli\volt}-\qty{1.5}{\volt} \\
    FWHM                &       \qty{120}{\pico\second} \\
\hline
\end{tabular}
\caption{Requerimientos del generador de pulsos.}
\label{tab:pulser_requirements}
\end{table}
